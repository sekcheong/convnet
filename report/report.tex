\documentclass{article}
\usepackage{amsmath}
\usepackage{graphicx}
\usepackage{hyperref}

\hypersetup{
  colorlinks, linkcolor=red
}

\begin{document}


\title{CS838 Lab 3}
\author{Sek Cheong}
\maketitle

%\begin{abstract}
%The abstract text goes here.
%\end{abstract}

\section{Introduction}
Your introduction

\section{The Data Set}
Your data set

\section{The Experiment}
Your experiments

\begin{equation}
\epsilon = accuracy(tune,epoch-1) - accuracy(tune, epoch) >= 0.001
\end{equation}



\subsection{Unregularized}
Subsection

\begin{equation*}
HU=9, \eta=0.01, \epsilon=0.001, \alpha=0, \lambda=0, MaxEpoch=100
\end{equation*}



\subsection{Learning Rate}

sub section



\subsection{Number of Hidden Units}

subsection
\begin{center}
 \includegraphics[width=3.0in]{hiddenunits}
\end{center}



\subsection{Momentum}
subsection

\begin{center}
 \includegraphics[width=3.0in]{momentum}
\end{center}



\subsection{Weight Decay}
subsection

\begin{center}
 \includegraphics[width=3.0in]{weightdecay}
\end{center}

%\begin{figure}
%    \centering
%    \includegraphics[width=2.0in]{learningrate}
%    \caption{Simulation Results}
%    \label{simulationfigure}
%\end{figure}



\section{Conclusion}
conclusion
\begin{equation*}
HU=9, \eta=0.009, \epsilon=0.001, \alpha=0.90, \lambda=0.0006,  MaxEpoch=150
\end{equation*}
A network with properly tuned learning rate, momentum, and weight decay could significantly out perform networks without any regularization. The weight decay did not seem to be very useful in out experiment.



\end{document}